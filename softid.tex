\documentclass[11pt,a4paper]{ivoa}
\input tthdefs

\lstset{flexiblecolumns=true,numberstyle=\small,showstringspaces=False,
  identifierstyle=\texttt,defaultdialect=[latex]tex,language=tex}

\title{Operational Identification of Software Components in the Virtual
Observatory}

% see ivoatexDoc for what group names to use here
\ivoagroup{Operations}

\author[https://wiki.ivoa.net/twiki/bin/view/IVOA/WebHome?topic=MarkusDemleitner]{Markus
Demleitner}
\author[https://wiki.ivoa.net/twiki/bin/view/IVOA/MarkusDemleitner?topic=MarkTaylor]{Mark
Taylor}

\editor{Markus Demleitner}

% \previousversion[????URL????]{????Concise Document Label????}
\previousversion{This is the first public release}

\newcommand{\headername}[1]{{\tt #1}}

\begin{document}
\begin{abstract}
In the Virtual Observatory, software often talks to other software.
During development, while gathering statistics (e.g., ignoring
validators), or in operations (e.g., enabling workarounds for known
bugs), it is often beneficial to know something about the
counterpart.  This note collects recommended practices on both the
client and the server side for HTTP-based protocols.
These recommendations make use of the existing standard HTTP headers
\headername{User-Agent} for client requests and
\headername{Server} for server responses.
\end{abstract}

\section*{Acknowledgement}

The bulk of this material was taken from a page on the IVOA
wiki\footnote{\url{https://wiki.ivoa.net/twiki/bin/view/IVOA/UserAgentUsage}}
that resulted from a discussion at the 2018 College Park Interop (client
side) and a
talk\footnote{\url{https://wiki.ivoa.net/internal/IVOA/InterOpOct2019Ops/serversoftware.pdf}}
at the 2019 Groningen Interop (server side).


\section*{Conformance-related definitions}

The words ``MUST'', ``SHALL'', ``SHOULD'', ``MAY'', ``RECOMMENDED'', and
``OPTIONAL'' (in upper or lower case) used in this document are to be
interpreted as described in IETF standard RFC2119 \citep{std:RFC2119}.

The \emph{Virtual Observatory (VO)} is a
general term for a collection of federated resources that can be used
to conduct astronomical research, education, and outreach.
The \href{http://www.ivoa.net}{International
Virtual Observatory Alliance (IVOA)} is a global
collaboration of separately funded projects to develop standards and
infrastructure that enable VO applications.


\section{Introduction}

Very early on in the construction of client-server architectures it was
found that it is useful to have mechanisms for 
discovering which software runs
at the other side of a connection, rather typically to aid in debugging.
In particular, HTTP, which is the basis of many of the VO's protocols
\citep{std:HTTP}, specifies the request header \headername{User-Agent} that
identifies the client software (``for statistical purposes, the tracing
of protocol violations, and automated recognition of user agents for the
sake of tailoring responses to avoid particular user agent
limitations'') and the response header \headername{Server}.

These general rules were found insufficient for various purposes in the
Virtual Observatory.  For instance, use case~\ref{uc:stats} was brought
up at the College Park Interop in 2018.  Also, it was found that most
servers did not identify the relevant protocol implementations; the
server headers rather identified
libraries the services were built with or even reverse proxies they
ran behind -- information that is usually not useful when diagnosing
operational problems in the VO, which mostly originate in the higher
layers of request handling.

This note gives recommendations on how to make the header fields more
useful for VO operations.  While strictly only applicable to HTTP-based
protocols, we believe similar practices should be implemented where, as
with most VOEvent transports, communication mechanisms are not built on
top of HTTP.


\subsection{Use Cases}

\subsubsection{Recognising Maintenance Queries}
\label{uc:stats}

In the VO, several clients connect to services for operational purposes,
for instance in order to perform service validation or monitoring. If
service providers are gathering statistics on service usage, they may
wish to distinguish these different classes of requests from requests
presumably coming from science users.

\subsubsection{Dropping Workarounds}
\label{uc:global}

Clients occasionally work around bugs in server software; these
workarounds over time are a maintenance liability, and hence it is
advantageous to drop them when they are no longer needed.  To establish
when a workaround can safely be dropped, developers need a simple way to
enumerate which server software is still running in the VO.

Conversely, a server may contain workarounds for client bugs.  Again,
being able to find out whether code implementing these can be safely
dropped without adverse consequences on users is obviously beneficial.

\subsubsection{Client Snooping}

A server may offer experimental or advanced features to clients it knows
(though such use should in general be frowned upon, as it violates the
spirit of interoperability).  Similarly, a client might under certain
circumstances want to enable or disable certain behaviour when realising
it is communicating with a known server component.

\subsubsection{Debugging}

For diagnosing failures, it is often useful to know which components are
part of a communication leading up the failure.  This is in particular
true to avoid unnecessary analysis when known-obsolete or highly
experimental components are the root cause.

\subsubsection{Notifications}

As part of a responsible disclosure of a software weakness (or simply a
request for a software update), server developers might want to contact
deployers of vulnerable or otherwise broken software.

\subsection{Security and Privacy Considerations}

Several guidelines on IT security discourage giving details on the
software that drives a certain site in order to not give attackers
information that might be useful in an attack.

Following the practices proposed here will, indeed, weaken the
``security by obscurity'' put forward in these treatments; on the other
hand, when, as is the case in the VO, attackers only have 
to scan perhaps several hundred URLs,
relying on security by obscurity does not seem a promising policy.

On the other
hand, in the VO, where software providers rather typically are members
of the community, and given the Registry which allows rapid discovery of
active services, these software providers can contact
operators of vulnerable services given sufficiently precise software
identification even before a vulnerability is disclosed.

The identification of the client has fewer security implications, as it
seems unlikely that rogue services could be aided by information on the
client version when they target clients.

Software identification does play a role in user privacy; user agents
are regularly employed in user tracking on the WWW.  While, presumably,
the generally non-profit operators in the VO will not use such data to
significantly violate their users' privacy, client authors may want to
give users the possibility to somewhat reduce the information content of
the headers proposed here.

On the other hand, the mechanisms proposed here are most relevant for
automated clients for which there usually are no privacy concerns.

Note also that the recommendations presented in this Note
have only incremental impacts on security and privacy;
the standard server and client header usage on which they
build already have implications in these areas.


\section{Client Identification}
\label{sect:client}

\subsection{User-Agent Header Standard Usage}
\label{sect:user-agent}

The HTTP \headername{User-Agent} header may be used by clients
to identify their
nature or origin. The definition and usage of this header is described
in RFC 2616 \citep{std:HTTP} section 14.43, with additional text on
syntax in sections 3.8 and 2.2.
The definition was updated in RFC 7231 \citep{std:RFC7231}, section 5.5.3.
The basic rule is that the content of
this field should consist of a sequence of tokens, where each token is
either a product name (with an optional version indicator), or a
free-text comment enclosed in parentheses. Formally (BNF from RFC 2616):

\begin{verbatim}
User-Agent        = "User-Agent" ":" 1*( product | comment )
product           = token ["/" product-version]
product-version   = token
comment           = "(" *( ctext | quoted-pair | comment ) ")"
ctext             = <any TEXT excluding "(" and ")">
token             = 1*<any CHAR except CTLs or separators>
quoted-pair       = "\" CHAR
\end{verbatim}

Additional rules and conventions are that more-significant tokens should
appear earlier in the sequence, and that the content should be ``short
and to the point''.

\subsection{User-Agent Header IVOA Recommendations}

The Operations IG endorses and encourages use of these standard
rules concerning the \headername{User-Agent} header,
and adds a further convention, which does not
conflict with the above rules: clients whose primary purpose
is \emph{operational}, as opposed to \emph{scientific},
should indicate that purpose by including a
comment token of the form 
$$\hbox{\verb|(IVOA-<op-purpose> <optional-extra-text>)|.}$$

Suggested {\tt op-purpose} values are currently:

\begin{description}
\item[test]
The purpose of the access is to test whether services are available
(monitoring) and/or standards-compliant (validation); at this point,
no good reason to separate the different cases was identified.
\item[copy] 
The purpose of the access is to replicate (parts of) the content
published through
the service, be it for aggregation (harvesting) or re-publication
(mirroring).
\end{description}

This list may evolve in the future; extensions should be proposed on 
the ops@ivoa.net mailing list. Custom {\tt op-purpose} values are permitted.
Case is significant in {\tt op-purpose} values and its ``{\tt IVOA-}'' prefix.

The \verb|<optional-extra-text>| could be used to indicate a URL at which
more information about the client, or perhaps about the results it is
gathering from the current request, can be found.
However, in accordance with the injunction from RFC 7231
{\em ``A user agent SHOULD NOT generate a User-Agent field containing
needlessly fine-grained detail''},
such additional text should be added only when it serves a real purpose.

Formally:

\begin{verbatim}
ivoa-comment  = "(IVOA-" op-purpose *( 
    ctext | quoted-pair | comment ) ")"
op-purpose   = "test" | "copy" | token
\end{verbatim}

Tokens of the form \verb|ivoa-comment| should not appear in the
\headername{User-Agent} field
if the request is a ``normal'' user science query. There
are obviously grey areas between operational and science requests; this
convention does not attempt to provide a rigid definition of these
categories.

This arrangement allows service operators to test in their logs for
\headername{User-Agent} values
whose content matches the sequence ``\verb|(IVOA-|'', or
perhaps ``\verb|(IVOA-test|'', and adjust their usage statistics
appropriately. Note, however, that it is not feasible to force operational
clients to follow this convention, so service operators will still need
to be careful in analysing their usage statistics.

\subsection{Examples}

A science query from the STILTS tapquery TAP client might contain the
HTTP header
\begin{verbatim}
User-Agent: STILTS/3.1-4 Java/1.8.0_181
\end{verbatim}
while a query from the STILTS taplint TAP service validator might
contain the header
\begin{verbatim}
User-Agent: STILTS/3.1-4 (IVOA-test) Java/1.8.0_181
\end{verbatim}
or maybe 
\iftth
\begin{verbatim}
User-Agent: STILTS/3.1-4 (IVOA-test http://validators.org/results) Java/1.8.0_181
\end{verbatim}
\else
(line break for typographic reasons)
\begin{verbatim}
User-Agent: STILTS/3.1-4 (IVOA-test
            http://validators.org/results) Java/1.8.0_181
\end{verbatim}
\fi

The server identification example, tapstat.py, discussed in
section~\ref{sect:server} illustrates one way to add such headers using
Python's built-in urllib.

In Java, an application can be configured to add tokens in this way
to the \headername{User-Agent} value for all HTTP client requests
by setting the {\tt http.agent} system property.


\section{Server Identification}
\label{sect:server}

\subsection{Server Header Standard Usage}

The HTTP \headername{Server} header may be used by servers to
identify their implementation software.
It is defined by RFC 2616 section 14.38,
updated by RFC 7231 section 7.4.2.
The header value syntax is identical to that for the
\headername{User-Agent} header described in section \ref{sect:user-agent}.

\subsection{Server Header IVOA Recommendations}

We recommend using the \headername{Server} header
in accordance with standard usage,
but VO servers should where possible include product tokens
for the VO software actually processing the request.
As on the client side, list VO components in front of
more generic HTTP server software.

\subsection{Examples}

A server running DaCHS might take steps to issue HTTP responses
containing the header:
\begin{verbatim}
DaCHS/2.2.1 twistedWeb/18.9.0
\end{verbatim}
which would be preferred over the basic
\begin{verbatim}
TwistedWeb/18.9.0
\end{verbatim}
which the server infrastructure might provide by default.

This note comes with an example programme obtaining global server
statistics for registered TAP
services\footnote{\auxiliaryurl{tapstats.py}; for the S*AP protocols,
some URL heuristics might be necessary in order to achieve true server
enumeration.}.

\subsection{Notes}

Use case~\ref{uc:global} might appear to require Registry support for
server identification to enable queries like ``give me a list of all
server software in use in the VO'' or ``which operators run version 21.2
of software X?''  However, given that it is unlikely that the VO will
ever host more than a few hundred distinct servers of a given type
(under the assumption that each piece of software on a large data centre
will serve many different resources), the use cases for global server
identification can probably be satisfied by running one request each
against these servers, access URLs for which can readily be discovered
in the Registry as it is.  

\appendix
\section{Changes from Previous Versions}

No previous versions yet.  
% these would be subsections "Changes from v. WD-..."
% Use itemize environments.


\bibliography{ivoatex/ivoabib,ivoatex/docrepo}


\end{document}
