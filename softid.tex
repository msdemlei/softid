\documentclass[11pt,a4paper]{ivoa}
\input tthdefs

\title{Operational Identification of Software Components in the Virtual
Observatory}

% see ivoatexDoc for what group names to use here
\ivoagroup{Ops}

\author[https://wiki.ivoa.net/twiki/bin/view/IVOA/WebHome?topic=MarkusDemleitner]{Markus
Demleitner}
\author[https://wiki.ivoa.net/twiki/bin/view/IVOA/MarkusDemleitner?topic=MarkTaylor]{Mark
Taylor}

\editor{Markus Demleitner}

% \previousversion[????URL????]{????Concise Document Label????}
\previousversion{This is the first public release}
       

\begin{document}
\begin{abstract}
In the Virtual Observatory, software often talks to other software, and
while doing it, it is often beneficial to know something about the
counterpart, be it for statistics (e.g., when ignoring validators) or
for operations (e.g., when enabling workarounds, or when such
workarounds can be dropped because components requiring them are no
longer in use).  This note collects recommended practices on both the
client and the server side.
\end{abstract}


\section*{Conformance-related definitions}

The words ``MUST'', ``SHALL'', ``SHOULD'', ``MAY'', ``RECOMMENDED'', and
``OPTIONAL'' (in upper or lower case) used in this document are to be
interpreted as described in IETF standard RFC2119 \citep{std:RFC2119}.

The \emph{Virtual Observatory (VO)} is a
general term for a collection of federated resources that can be used
to conduct astronomical research, education, and outreach.
The \href{http://www.ivoa.net}{International
Virtual Observatory Alliance (IVOA)} is a global
collaboration of separately funded projects to develop standards and
infrastructure that enable VO applications.


\section{Introduction}




\appendix
\section{Changes from Previous Versions}

No previous versions yet.  
% these would be subsections "Changes from v. WD-..."
% Use itemize environments.


\bibliography{ivoatex/ivoabib,ivoatex/docrepo}


\end{document}
