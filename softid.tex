\documentclass[11pt,a4paper]{ivoa}
\input tthdefs

\title{Operational Identification of Software Components in the Virtual
Observatory}

% see ivoatexDoc for what group names to use here
\ivoagroup{Ops}

\author[https://wiki.ivoa.net/twiki/bin/view/IVOA/WebHome?topic=MarkusDemleitner]{Markus
Demleitner}
\author[https://wiki.ivoa.net/twiki/bin/view/IVOA/MarkusDemleitner?topic=MarkTaylor]{Mark
Taylor}

\editor{Markus Demleitner}

% \previousversion[????URL????]{????Concise Document Label????}
\previousversion{This is the first public release}
       

\begin{document}
\begin{abstract}
In the Virtual Observatory, software often talks to other software, and
while doing it, it is often beneficial to know something about the
counterpart, be it for statistics (e.g., when ignoring validators) or
for operations (e.g., when enabling workarounds, or when such
workarounds can be dropped because components requiring them are no
longer in use).  This note collects recommended practices on both the
client and the server side.
\end{abstract}

\section*{Acknowledgement}

The bulk of this material was taken from a page on the IVOA
wiki\footnote{\url{https://wiki.ivoa.net/twiki/bin/view/IVOA/UserAgentUsage}}
that resulted from a discussion at the 2018 College Park Interop (client
side) and a
talk\footnote{\url{https://wiki.ivoa.net/internal/IVOA/InterOpOct2019Ops/serversoftware.pdf}}
at the 2019 Groningen Interop (server side).


\section*{Conformance-related definitions}

The words ``MUST'', ``SHALL'', ``SHOULD'', ``MAY'', ``RECOMMENDED'', and
``OPTIONAL'' (in upper or lower case) used in this document are to be
interpreted as described in IETF standard RFC2119 \citep{std:RFC2119}.

The \emph{Virtual Observatory (VO)} is a
general term for a collection of federated resources that can be used
to conduct astronomical research, education, and outreach.
The \href{http://www.ivoa.net}{International
Virtual Observatory Alliance (IVOA)} is a global
collaboration of separately funded projects to develop standards and
infrastructure that enable VO applications.


\section{Introduction}


\subsection{Use Cases}

\subsubsection{Recognising Maintenance Queries}

In the VO, several clients connect services for operational purposes,
for instance in order to perform service validation or monitoring. If
service providers are gathering statistics on service usage, they may
wish to distinguish these different classes of request.

\subsubsection{Dropping Workarounds}

Client occasinally work around bugs in server software; these
workarounds over time are a maintenance liability, and hence it is
advantageous to drop them when they are no longer needed.  To find this
out, a simple way to enumerate server software running on the VO.

Conversely, a server may contain workarounds for client bugs.  Again,
being able to find out whether code implementing these can be safely
dropped without adverse consequences on clients actually in use has a
clear benefit.

\subsubsection{Client Snooping}

Servers may offer experimental or advanced features to clients it knows
(though such use should in general be frowned upon, as it violates the
spirit of interoperability).  Similarly, clients might under certain
circumstances want to enable or disable certain behaviour when realising
they are communicating with a known server component.

\subsubsection{Notifications}

As part of a responsible disclosure of a software weakness (or simply a
request for a software update), server developers might want to contact
deployers of vulnerable or otherwise broken software.


\appendix
\section{Changes from Previous Versions}

No previous versions yet.  
% these would be subsections "Changes from v. WD-..."
% Use itemize environments.


\bibliography{ivoatex/ivoabib,ivoatex/docrepo}


\end{document}
